Tässä työssä tarkasteltiin suorituskyvyntestaustyökalun valintaa web-sovelluspalveluille. Suorituskyvyntestaustyökaluja on hyvin monia, joten voi olla haastavaa vertailla eri työkaluja ja valita sopivin vaihtoehto. Asiaan perehdyttiin kirjallisuuskatsauksen keinoin ja pyrittiin löytämään vastaukset tutkimuskysymyksiin: 

\begin{enumerate}
    \item Miten valitaan oikea suorituskyvyntestaustyökalu web-sovelluspalveluiden testaamiseen?
    \item Mitä tekijöitä pitää ottaa huomioon, kun vertaillaan eri suorituskyvyntestaustyökaluja?
\end{enumerate}

Kirjallisuuskatsauksen perusteella suorituskyvyntestaustyökalujen valintaa pohditaan erilaisten tekijöiden avulla. Tästä syystä tämän työn ydintulokset koostuivat näiden tekijöiden löytämisestä ja analysoimisesta. Löydetyt tekijät jaettiin selkeyden vuoksi kahteen kategoriaan, organisatorisiin tekijöihin ja työkalukohtaisiin tekijöihin. Organisatoristen tekijöiden \hyperref[sec:organisatorisettekijät]{alaluvussa} perehdyttiin tekijöihin, jotka liittyivät suorituskyvyntestaustyökalujen \hyperref[ssec:hinta]{hinnoitteluun}, \hyperref[ssec:tarpeensuuruus]{tarpeen suuruuteen} ja \hyperref[ssec:tuki]{saatavilla olevaan tukeen}. Työkalukohtaisten tekijöiden \hyperref[sec:työkalukohtaisettekijät]{alaluvussa} perehdyttiin \hyperref[ssec:tarvittavatestaustyyppi]{tarvittavaan testaustyyppiin}, \hyperref[ssec:yhteensopivuus]{yhteensopivuuteen} ja \hyperref[ssec:käytettävyys]{käytettävyyteen}. Löydetyt tekijät, ja täten koko työn tulokset, olivat kattavat ja onnistuivat vastaamaan tämän työn tutkimuskysymyksiin hyvin. Työkalun valintaan vaikuttavia tekijöitä on kuitenkin monia, ja tästä syystä niitä kannattaa tarkastella tilanteen mukaan. Oma tarve saattaa vaatia jotain erikoistunutta ominaisuutta, jota ei mainita missään muiden tekemissä oppaissa. Näistä syistä voidaan todeta, että mitään yleispätevää ohjetta suorituskyvyntestaustyökalun valintaan ei voida mielekkäästi muotoilla.

Tässä työssä suorituskyvyntestaustyökalun valintaan liittyvä pohdinta oli rajattu web-so\-vel\-lus\-pal\-ve\-lui\-hin. Rajaus tehtiin, koska suorituskyvyntestaus, ja siihen käytettävät työkalut, on kokonaisuudessaan liian laaja aihe tämän työn käsiteltäväksi. Rajausten puitteissa karsittiin pois myös lähteitä, joissa työkaluja vertailtiin eri tekijöihin perehtymisen sijaan joillain täysin muilla lähestymistavoilla. Rajausten ulkopuolelle jätetyistä aiheista löytyy vielä mielenkiintoisia jatkotutkimusaiheita. Olisi mielenkiintoista esimerkiksi vertailla eräässä tutkimuksessa kehitettyä matemaattista suorituskyvyntestaustyökalun valintamallia perinteiseen tekijöiden pohtimiseen. Olisi myös mahdollista perehtyä täysin oman työkalun kehittämiseen ja vertailla sen toteuttamiskelpoisuutta ja mahdollisia kustannuksia valmiin työkalun käyttämiseen.
\looseness=-1
