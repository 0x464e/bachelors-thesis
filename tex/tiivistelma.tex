Suorituskyvyntestaustyökalut (engl. performance testing tool) ovat testiautomaatioon liittyviä työkaluja, joilla voidaan testata järjestelmän suorituskykyä. Suorituskyvyntestaustyökaluja on monia erilaisia ja niistä pitää osata valita oikea omiin tarpeisiin. Erilaisia ohjelmistoja, joilla on tarvetta suorituskyvyntestaukseen, on erittäin monia. Tästä syytä tämän työn aihe on rajattu web-sovelluspalveluiden suorituskyvyntestaamiseen. Miten valitaan oikea suorituskyvyntestaustyökalu web-sovelluspalveluiden testaamiseen? Mitä tekijöitä pitää ottaa huomioon, kun vertaillaan eri työkaluja?


Web-sovelluspalvelut (engl. web service) ovat ohjelmistojärjestelmiä, jotka mahdollistavat tietokoneiden keskeisen vuorovaikutuksen. Toisien sanottuna ne ovat hyvin keskeinen osa nykymaailman IT-infrastruktuuria. Tästä syystä niiden moitteettomaan toimimiseen pyritään ohjelmistotestauksella, jonka yksi tärkeä vaihe on suorituskyvyntestaus. Suorituskyky kertoo ohjelman kyvystä toimia nopeasti ajan suhteen ja sen kyvystä käyttää saatavilla olevia resursseja tehokkaasti. Suorituskykyä testataan erikoistuneilla suorituskyvyntestaustyökaluilla.


Työ suoritettiin kirjallisuuskatsauksena, joten käytettävä aineisto saatiin suorittamalla hakuja eri hakupalveluilla. Tehdyt tiedonhaut voidaan jakaa kahteen kategoriaan. Ensimmäiseen kategoriaan kuuluu haut, joilla etsittiin suoraan vastauksia tämän työn tutkimuskysymyksiin. Toinen kategoria sisältää haut, joilla perusteltiin työn taustatietoja, tai löydettyjen suorituskyvyntestaustyökalujen valintaan liittyvien tekijöiden tärkeyttä. Kirjallisuuskatsauksessa suosittiin tieteellisiä lähteitä ja pyrittiin löytämään mielipiteiden sijaan tieteellisiä perusteluja.


Työn tulokset muodostuvat kirjallisuuskatsauksen avulla löydettyjen valintatekijöiden käsittelystä. Eri valintatekijät jaetaan selkeyden vuoksi kahteen kategoriaan, organisatorisiin tekijöihin ja työkalukohtaisiin tekijöihin. Nämä kaksi kategoriaa jaetaan molemmat vielä kolmeen erikseen käsiteltävään alakategoriaan. Organisatorisissa tekijöissä käsitellään suorituskyvyntestaustyökalujen rahallisia kustannuksia, käyttötarpeen suuruutta ja saatavilla olevaa tukea. Työkalukohtaisissa tekijöissä perehdytään tarvittavaan testaustyyppiin, yhteensopivuuteen ja käytettävyyteen. Edellä mainittuja tieteellisistä lähteistä löydettyjä valintatekijöitä vertaillaan internetistä löytyviin alan ammattilaisten henkilökohtaisiin mielipiteisiin tai oppaisiin. Vertailun perusteella saadaan selville, että tämän työn tulokset ovat kattavat ja käsittelevät suorituskyvyntestaustyökalujen valintaan liittyviä tekijöitä laajasti.
