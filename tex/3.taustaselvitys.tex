Tässä luvussa käydään läpi pohjatiedot työn aiheen ymmärtämiseksi. Ensin perehdytään web-sovelluspalveluihin. Tästä siirrytään suorituskyvyn ymmärtämiseen käsitteenä ja sen testaamiseen. Viimeiseksi kerrotaan taustatiedot itse suorituskyvyntestaustyökaluista.

\section{Web-sovelluspalvelut}
\label{sec:web-sovelluspalvelut}
Termi web-sovelluspalvelu, tai varsinkin sen arkikielinen nimitys web-palvelu, sekoitetaan usein muihin samankaltaisiin termeihin, kuten verkkopalveluihin. Termin määritelmä on kuitenkin \acrlong{w3c}in (\acrshort{w3c}) mukaan ohjelmistojärjestelmä, joka mahdollistaa yhteensopivien tietokoneiden vuorovaikutuksen jonkin tietokoneverkon yli. \parencite{w3c}

Web-sovelluspalveluilla on jokin rajapinta, joka mahdollistaa vuorovaikutuksen muiden järjestelmien kanssa. Vuorovaikutus tapahtuu käyttäen jotain protokollaa. Useimmissa tapauksissa protokollana toimii \acrshort{http} (engl. \acrlong{http}), mutta protokolla voi olla myös mikä tahansa muu. \parencite{w3c} Web-sovelluspalvelun tarjoama rajapinta, ja standardisoidun protokollan käyttö, mahdollistaa ohjelmointikielestä ja alustasta riippumattoman kommunikaation järjestelmien välillä. 

Muita esimerkkejä usein käytetyistä protokollista web-sovelluspalveluiden kanssa ovat esimerkiksi \acrshort{soap} (engl. \acrlong{soap}), \acrshort{rest} (engl. \acrlong{rest}), \acrshort{mqtt} (engl. \acrlong{mqtt}) ja \acrshort{coap} (engl. \acrlong{coap}). \acrshort{rest} ei ole käytännössä protokolla, vaan arkkitehtoninen tyyli toteuttaa sovellus. \acrshort{rest} kuitenkin rinnastetaan usein \acrshort{soap}:iin, koska ne ovat molemmat hyvin yleisiä tapoja toteuttaa web-sovelluspalvelu. Tästä syystä \acrshort{rest} esitellään tässä alaluvussa. \acrshort{mqtt} ja \acrshort{coap} ovat yleisiä \acrshort{iot}-protokollia (engl. \acrlong{iot}) ja ne esitellään tässä alaluvussa yhtenä esimerkkinä monista mahdollisista erilaisista protokollista.

\acrshort{soap} on vanhempi protokolla, joka luotiin ennen \acrshort{rest}:iä. \acrshort{soap} on \acrshort{rest}:iä hitaampi ja raskaampi, koska jokainen \acrshort{soap}-viesti sisältää siirrettävän datan lisäksi paljon oheistietoa \acrshort{xml}-formaatissa (engl. \acrlong{xml} format). Tämä oheistieto muun muassa kuvailee sovellusta, jossa protokolla on käytössä. \acrshort{rest}:iä noudattavat viestit sen sijaan ovat hyvin minimaalisia, ja eivät sisällä mitään oheistietoa siirettävän datan lisäksi. Tämä voi olla sekä hyöty, että haitta, riippuen halutusta käyttötarkoituksesta. \parencite{SOAPvsREST} \acrshort{mqtt}-protokolla toimii \acrshort{tcp}:n (engl. \acrlong{tcp}) päällä, kun taas \acrshort{coap} toimii \acrshort{udp}:n (engl. \acrlong{udp}) päällä \parencite{MQTTvsCoAP}. Nämä kaksi tunnettua \acrshort{iot}-protokollaa toimivat oivana esimerkkinä siitä, miten protokollat voivat hyödyntää toisiaan ja toimia toistensa päällä. Aivan kuten \acrshort{soap} voi toimia esimerkiksi \acrshort{http}:n tai jopa \acrshort{smtp}:n (engl. \acrlong{smtp}) päällä. \acrshort{smtp} on yleisin protokolla, joka mahdollistaa sähköpostien lähettämisen ja vastaanottamisen. 

\section{Suorituskyvyn mittaaminen}
\label{sec:suorituskyky}
Suorituskyky kertoo ohjelman nopeudesta ajan suhteen ja siitä, miten tehokkaasti ohjelma käyttää resursseja. Suorituskykyä ei ole määritelty jollain tietyllä yksiselitteisellä ominaisuuksien joukolla, mutta sitä voidaan kuvata tietyillä mittareilla. Yleisiä mittareita ovat reagointikyky, vasteaika, suoritusteho, saatavuus, skaalautuvuus ja käyttöaste. \parencites[4]{WhatIsSoftwarePerformance}[2]{PerformanceTestingGuidanceForWebApplications}[2-3]{TheArtOfApplication} Reagointikyky kertoo, miten nopeasti ohjelma pystyy reagoimaan tiettyyn asiaan. Vasteaika kertoo, kuinka kauan tietyllä tehtävällä kestää läpäistä ohjelma. Suoritusteho kertoo, miten paljon tiettyjä tehtäviä ohjelma pystyy suorittamaan ajan hetkessä. Saatavuus kertoo, kuinka suuren osuuden ajasta ohjelma on valmiina ottamaan uuden tehtävän eikä ole jumissa muun tehtävän kanssa. Skaalautuvuus kertoo ohjelman toimintakyvystä, kun käsiteltävän tehtävien tai datan määrää kasvatetaan. Käyttöaste kertoo, kuinka suuren osan ajasta tietty ohjelman osa tekee töitä verrattuna muihin ohjelman osiin.

Edellisessä kappaleessa esitettiin vain teknillinen näkökulma suorituskyvyn määritelmälle. Suorituskyvyn määritelmä riippuu näkökulmasta. Tavalliselle loppukäyttäjälle hyvä suorituskyky tarkoittaa sitä, että ohjelma suorittaa annetut tehtävät ilman havaittavaa hidastumista tai muuta ärsytystä \parencite[1-2]{TheArtOfApplication}. Tämän kautta päästään kiinni suorituskyvyn testaamisen vaikeuteen. Miten voidaan testata teknillisesti vaikeaa asiaa, jonka onnistuminen on loppujen lopuksi kiinni vain ihmisen havaitsemasta lopputuloksesta? 

\section{Suorituskyvyntestaus}
\label{sec:suorituskyvyntestaus}
Ohjelmistotestaus on tärkeä osa ohjelmiston kehityksen elinkaarta ja suorituskyvyntestaus on tärkeä osa ohjelmistotestausta \parencite{ScrutinyOnVariousApproaches}. Suorituskykyä testaamalla voidaan muun muassa arvioida tuotteen valmiutta, arvostella suorituskykyä haluttujen kriteerien perusteella, etsiä suorituskykyongelmien lähteitä tai arvioida tuotteen suorituskykyä eri järjestelmissä tai konfiguraatioissa \parencite{PerformanceTestingGuidanceForWebApplications}. Hyvä suorituskyky on myös tärkeää rahallisesti. Suuryritys Amazon löysi tutkimuksessa, että jokainen sadan millisekunnin viive tuotti yhden prosentin häviön vuotuisissa tuotoissa. Häviö tuotoissa perustui siihen, että käyttäjät eivät jaksaneet odottaa sivun latautumista, vaan sulkivat sen ja koettivat jotain muuta sivua hitaasti latautuvan sivun sijaan. \parencite{AmazonRevenue}

Toisin kuin monissa muissa testausmenetelmissä, suorituskyvyn testauksessa testataan ohjelman ei-toiminnallisia vaatimuksia. Luvussa \ref{sec:suorituskyky} mainittuja mittareita voidaan pitää ei-toiminnallisina vaatimuksia. Esimerkki toiminnallisen ja ei-toiminnallisen vaatimuksen erosta voisi olla viestintäpalvelu, jossa lähetetään viesti. Toiminnallinen vaatimus voisi olla, että viestin pystyy lähettämään ja vastaanottaja saa sen viestin. Ei-toiminnallinen vaatimus voisi olla, että järjestelmä pystyy skaalautumaan 10 000 yhtäaikaisen käyttäjän viestittelyyn.

Suorituskyvyntestaus on yleisnimitys monelle eri suorituskyvyntestaustyypille. Eri suorituskyvyntestaustyyppejä ovat muun muassa rasitustestaus (engl. stress testing), kuorman testaus (engl. load testing), vakaustestaus (engl. soak testing tai stability testing), skaalautuvuuden testaus (engl. scalability testing) ja piikkitestaus (engl. spike testing). Rasitustestauksessa rasitetaan ohjelmaa niin paljon, kunnes jokin menee rikki. Näin saadaan selville rajat, joissa ohjelma voi toimia. Kuorman testauksessa testataan, miten ohjelma toimii erikokoisten kuormien alaisena. Vakaustestauksessa testataan ohjelman vakautta ja toimintakykyä pitkäaikaisen suuren kuorman alaisena. Skaalautuvuuden testauksessa testataan ohjelman kykyä skaalautua kasvavaan käyttömäärään. Piikkitestauksessa testataan ohjelman kykyä suoriutua yllättävästä hetkellisestä suuresta kuormituksesta. \parencite{SoftwarePerformanceTesting, WebPerformanceTestingTools}

Kuvasta \ref{fig:agilequadrants} löytyvä Crispinin ja Gregoryn yleistämä ketterän testauksen neljän kvadrantin kuva vetää yhteen tässä alaluvussa käsitellyt asiat ja antaa oivan siirtymän suorituskyvyntestaustyökalujen käsittelyyn. Nelikentän oikea puoli, joka sisältää tuotetta kritisoivia testejä, liittyy vahvasti tämän työn aiheeseen. Oikealta puolelta löytyy monia tekijöitä, joita huomioidaan luvussa \ref{ch:työkalunvalinta}, kun pohditaan suorituskyvyntestaustyökalun valintaan liittyviä tekijöitä. Myös itse suorituskyvyntestaus löytyy nelikentän neljännestä kvadrantista, jonka kerrotaan sisältävän tuotetta kritisoivia ja teknologiaan keskittyviä testejä, joihin käytetään työkaluja. Nelikentän vasemmalla puolella olevat testit eivät suoraan liity tämän työn aiheeseen, koska ne ovat ohjelmistotestausprosessissa paljon aiempana kuin suorituskyvyntestaus.  

\begin{figure}[H]
\centering
\pdftooltip{\includesvg[width=.85\textwidth]{figures/kandi-testing-quadrants.svg}}
{Kuva. Neljä kvadranttia, joissa listataan ketterän kehityksen eri testausvaiheita. Suorituskykytestaus löytyy neljännestä kvadrantista}
\caption{Ketterän kehityksen testausprosessin neljä kvadranttia. Kuva mukailtu lähteestä \cite[97]{AgileTestingAPracticalGuide}.}
\label{fig:agilequadrants}
\end{figure}



\section{Suorituskyvyntestaustyökalut}
\label{sec:suorituskyvyntestaustyökalut}
Suorituskyvyntestaustyökalut ovat tietokonesovelluksia, joiden avulla voidaan suorittaa automatisoituja suorituskyvyntestejä. Suorituskyvyntestaustyökaluja on todella monia, ja niitä on ollut olemassa jo yli 30 vuotta. Eri työkaluja voidaan käyttää moniin erityyppisiin suorituskyvyntesteihin. Varsinkin web-sovelluksien testaamiseen työkaluja on monia. Kun siirrytään web-sovelluksien testauksesta muunlaisiin sovelluksiin, kuten esimerkiksi sulautettuihin järjestelmiin tai peleihin, sopivien suorituskyvyntestaustyökalujen määrä vähenee huomattavasti. \parencite[11-12]{TheArtOfApplication}

Suorituskyvyntestaustyökalujen tärkeimmät ominaisuudet koostuvat usein neljästä osasta, testien luonti, testien hallinta, testikuorman luonti ja tulosten analysointi \parencite[12]{TheArtOfApplication}. Eri työkaluissa nämä eri ydintoiminnot on toteutettu monilla eri tavoilla ja eri työkalut pystyvät erilaisiin toimintoihin. Testien luonti voi olla esimerkiksi koodin kirjoittamista, käyttöliittymässä kenttien täyttämistä tai toimintojen nauhoittamista. Testien hallinta sisältää testitapausten eri parametrien ja asetuksien säätämisen, sekä itse testien ajamisen. Testikuorman luontiin vaikuttaa työkalun kyky luoda erilaisia testikuormia. Esimerkiksi halutaan luoda kuorma, joka simuloi piikkitestausta, jossa 500 käyttäjää yrittää suorittaa jonkin toiminnon yhtä aikaa. Tulosten analysointiin liittyy toiminnot, joilla työkalu pystyy esittämään keräämänsä tulokset. Tähän voi liittyä erilaisia visualisointeja erilaisten diagrammien muodossa, joista käyttäjä voi tehdä itse johtopäätöksiä. \parencite{ComparativeAnalysisOfWeb} Näiden neljän ydintoiminnon lisäksi työkalut voivat sisältää myös muita moduuleja. Nämä perustoimintojen ulkopuolella olevat lisämoduulit usein täydentävät perustoimintojen ominaisuuksia. \parencite[13]{TheArtOfApplication} Lisämoduuli voisi olla esimerkiksi ohjelmointirajapinta (engl. \acrlong{api}, \acrshort{api}), jonka avulla työkalu voidaan liittää muihin järjestelmiin.

Perimmäinen syy työkalujen käyttöön testauksessa on testien automatisointi. Testien automatisoinnilla on monia perinteisiä hyvin tunnettuja etuja manuaaliseen testaukseen nähden, kuten esimerkiksi testauksen nopeuttaminen tai rahan säästäminen \parencite[528]{AgileTestingAPracticalGuide}. Automatisoinnin tuomien perinteisten hyötyjen lisäksi suorituskyvyntestauksessa on erityisiä hyötyjä työkalujen käytöstä. Useimmat luvussa \ref{sec:suorituskyky} mainituista mittareista ovat hyvin vaikeita, tai jopa mahdottomia, mitata manuaalisesti. Esimerkiksi manuaalinen skaalautuvuuden testaaminen olisi hyvin vaikeaa, kun haluttaisiin kasvattaa yhtäaikaista käyttäjien määrää suureksi. Suorituskyvyntestaustyökalu sen sijaan voi simuloida hyvinkin suuria määriä yhtäaikaisia käyttäjiä ongelmitta.
