%\begin{itemize}
%    \item kirjallisuuskatsaus
%    \item mitä hakusanoi, miten keksi hakusanat
%    \item mistä etsittiin
%    \item miten paljon tuloksii
%    \item mil perusteel otettiin mukaan
%    \item rajausta julkaisuvuodes?
%    \item tänne se fuzzy?!
%\end{itemize}

Tämä työn on toteutettu kirjallisuuskatsauksena ja haetut lähteet voidaan jakaa karkeasti kahteen kategoriaan. Ensimmäiseksi haettiin lähteitä, jotka liittyvät suoraan tämän työn tutkimuskysymyksiin. Esimerkiksi lähteitä, joissa käsitellään suoraan tekijöitä, jotka vaikuttavat suorituskyvyntestaustyökalun valintaan. Toiseksi haettiin lähteitä, joilla perusteltiin työn taustatiedot tai tiettyjen työkalun valintaan liittyvien tekijöiden tärkeys. Esimerkiksi lähteitä, jotka kertovat mikä on \gls{web-sovelluspalvelu}n määritelmä, tai miksi visualisoinnit ovat tärkeitä käytettävyyden kannalta. Ensimmäiseen kategoriaan kuuluvia lähteitä etsittiin Andor ja Google Scholar -hakupalveluiden avulla. Toiseen kategoriaan kuuluvia termien määrittelyjä sisältäviä lähteitä etsittiin myös suoraan Google-haulla.

Suoraan tutkimuskysymyksiin liittyviä lähteitä etsittäessä hakusanat muodostettiin suoraan aiheen sanastosta, kuten esimerkiksi ''performance testing tool``. Taustatietojen tai tekijöiden tärkeyttä perustelevia lähteitä etsittäessä hakusanat muodostettiin, kun oli saatu selville mitä halutaan perustella. Esimerkiksi aiemmin löydetyissä tutkimustuloksissa todettiin, että käyttöliittymän käytettävyys on tärkeää, joten perustelun löytämiseksi hakusanaksi otettiin ''interface usability``. Tämän työn tiedonhakuprosessi on toistettavissa seuraavilla hakulausekkeilla:
\begin{itemize}
    \item ''performance testing tool`` OR ''load testing tool``
    \item ''performance testing tools``
    \item performance testing web
    \item performance testing web service
    \item ''agile testing``
    \item interface usability
    \item SOAP vs REST
    \item MQTT vs CoAP
\end{itemize}

Melkein joka hakua suoritettaessa käytettiin Andor-hakupalvelun toimintoa rajata tulokset vain vertaisarvioituihin julkaisuihin. Hakuja suoritettaessa saatiin vaihtelevia määriä hakutuloksia kymmenistä tuhansiin. Julkaisuvuotta ei rajattu hakuja tehdessä, mutta jos vastaan tuli samaa asiaa sisältäviä lähteitä, uudempaa julkaisua suosittiin. Kun haku tuotti vain noin 100 tulosta tai alle, oli helppoa käydä otsikot tai tiivistelmät läpi kaikista tuloksista ja poimia jatkotarkasteluun lupaavat julkaisut. Suoraan tämän työn tutkimuskysymyksiin liittyvien lähteiden mukaanottoehdot olit seuraavat:

\begin{itemize}
    \item Lähteessä kerrotaan suoraan suorituskyvyntestaustyökalujen valintaan liittyvistä tekijöistä tai lähteessä vertaillaan eri suorituskyvyntestaustyökaluja
    \item Lähde on vertaisarvioitu, tai muuten korkeasti arvostettu
    \item Lähteen aihe ei ole tämän työn rajausten ulkopuolella
\end{itemize}

Muuten korkeasti arvostetun lähteen ehdon toteutti esimerkiksi \citeauthor{TheArtOfApplication}in (\citeyear{TheArtOfApplication}) kirja, johon oli viitattu erittäin monessa julkaisussa. Tämän työn rajausten sisäpuolelle laskettiin lähteet, joissa tarkastellaan web-sovelluspalveluihin liittyvien sovellusten suorituskyvyntestaamista. Web-sovelluspalvelurajoite karsi pois lähteitä, joissa tarkasteltiin esimerkiksi robottien tai pelkkien algoritmien suorituskyvyntestausta. Rajausten ulkopuolelle jäi myös lähteet, joissa oli keksitty jokin aivan oma muista poikkeava järjestelmä suorituskyvyntestaustyökalujen vertailuun. Kyseisissä lähteissä oli kehitetty esimerkiksi täysin matemaattinen malli suorituskyvyntestaustyökalujen vertailuun. Kutenkin tämän työn aiheen paisumisen estämiseksi täytyi rajata pois lähteet, joissa työkalunvalintaan oli kehitetty jokin muu menetelmä, kuin eri valintaan vaikuttavien tekijöiden pohtiminen.

Taustatietojen tai tekijöiden tärkeyttä perustelevia lähteitä etsittäessä hakutuloksia oli tuhansia. Tästä syystä tuhansien lähteiden läpikäynnin sijaan suosittiin julkaisuja, joihin oli viitattu useasti. Näistä julkaisuista käytiin läpi lähdeluettelot ja pyrittiin pääsemään primäärilähteeseen. Google-haulla etsittyjä lähteitä valittaessa suosittiin monien eri dokumentaatiosivujen sijaan yhteen vetäviä julkaisuja. Yhteen vetäviä Google-haulla löydettyjä lähteitä viitatessa asioiden pätevyys tarkistettiin kuitenkin primäärilähteestä, kuten jonkin standardin omasta dokumentaatiosivusta, ennen kuin lähde valittiin mukaan.