Yhä digitalisoituvassa nykymaailmassa tietoteknologiatuotteiden ja -palvelujen ala on biljoonien eurojen arvoinen. Hyvän IT-infrastruktuurin ylläpitäminen on välttämätöntä niin yrityksille, valtion palveluille kuin voittoa tavoittelemattomille järjestöillekin. \parencite{ITbudgets} Eri IT-järjestelmiä yhdistää web-sovelluspalvut (engl. web service), joten niiden moitteeton toimiminen on kriittinen osa IT-infrastruktuuria. Web-sovelluspalveluja kehitettäessä niiden moitteettomaan toimimiseen pyritään ohjelmistotestauksen avulla. Suorituskyvyntestaus on tärkeä osa ohjelmistotestausta, jolla varmistetaan muun muassa ohjelman nopea ja tehokas toimiminen. Suorituskykyä testataan suorituskyvyntestaustyökaluilla, joita on olemassa hyvin monia erilaisia. Kuitenkin sovelluskehitysprojektia varten täytyy ostata valita tarpeisiin sopiva suorituskyvyntestaustyökalu. 

Tässä työssä perehdytään tekijöihin, jotka vaikuttavat suorituskyvyntestaustyökalun valintaan. Suorituskyky, ja sen testaus, on kokonaisuudessaan todella laaja aihealue. Tämän kaiken käsittely ei mahtuisi tähän työhön. Tästä syystä työn näkökulma on rajattu web-sovelluspalveluiden suorituskyvyn testaamiseen työkaluilla. Tämä työ vastaa kysymyksiin, miten valitaan oikea suorituskyvyntestaustyökalu web-sovelluspalveluiden testaamiseen ja mitä tekijöitä pitää ottaa huomioon, kun vertaillaan eri työkaluja. Aiheesta löytyy materiaalia enimmäkseen oppaina. Oppaissa harvoin on mitään lähteitä ja ne perustuvat vain alan ihmisten omiin kokemuksiin ja mielipiteisiin. Tästä syystä on mielenkiintoista suorittaa aiheesta kirjallisuuskatsaus ja löytää tieteellisiä perusteita eri tekijöille, jotka vaikuttavat suorituskyvyntestaustyökalun valintaan.

Luvussa \ref{ch:tutkimusmenetelmä} kuvataan tutkimusmenetelmä ja tiedonhakuprosessi. Tiedonhaunprosessista kerrotaan käytetyt hakusanat ja millä perusteella lähteitä otettiin mukaan työhön. Luvussa \ref{ch:taustaselvitys} perehdytään työn aiheen taustatietoihin. Taustatietoihin kuuluu keskeisten käsitteiden ymmärtäminen, suorituskyvyntestauksen tärkeyden sekä toimintaperiaatteen hahmottaminen ja suorituskyvyntestaustyökalujen toimintaperiaatteiden käsittäminen. Luvussa \ref{ch:työkalunvalinta} esitetään työn tulokset. Tulokset ovat suorituskyvyntestaustyökalujen valintaan liittyvien tekijöiden muodossa. Monia eri tekijöitä kuvaillaan esimerkkien avulla ja niiden tärkeyttä perustellaan kirjallisuuskatsauksella löydettyjen lähteiden avulla. Luvussa \ref{ch:tulostentarkastelu} tarkastellaan saatuja tuloksia ja vertaillaan niitä internetistä löytyviin oppaisiin. Tarkastelussa esitetään kritiikkiä tuloksia kohtaan ja pohditaan työssä olleita rajoitteita. Lopuksi luvussa \ref{ch:yhteenveto} vedetään yhteen työn aihe ja saadut tulokset.
