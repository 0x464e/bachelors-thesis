

%%%%% METADATA %%%%%
%% Always keep the following metadata up to date!
%% This is important for your PDF file to comply
%% to accessibility standards.
%% (And yes, this information must remain here,
%% before \documentclass[...]{...}.)

% \Title and \Language are mandatory,
% others desirable
% The appropriate Finnish language code is 'fi',
% UK English is en-UK
\begin{filecontents*}[overwrite]{\jobname.xmpdata}
\Title{Suorituskyvyntestaustyökalun valinta web-sovelluspalveluiden testaamiseen}
\Author{Otto}
\Language{fi}
\end{filecontents*}

\pdfminorversion=6

%%%%% PREAMBLE %%%%%

%%%%% Document class declaration.
% The possible optional arguments are
%   finnish - thesis in Finnish (default)
%   english - thesis in English
%   numeric - citations in numeric style (default)
%   authoryear - citations in author-year style
%   apa - citations in APA 7 (available only in English)
%   ieee - citations in IEEE style (available only in English)
%   draft - for faster non-final works, also skips images
%           (recommended, remove in final version)
%   programs - if you wish to display code snippets
% Example: \documentclass[english, authoryear]{tauthesis}
%          thesis in English with author-year citations
\documentclass{tauthesis}

% The glossaries package throws a warning:
% No language module detected for 'finnish'.
% You can safely ignore this. All other
% warnings should be taken care of!

%%%%% Your packages.
% Before adding packages, see if they can be found
% in tauthesis.cls already. If you're not sure that
% you need a certain package, don't include it in
% the document! This can dramatically reduce
% compilation time.

% Graphs
% \usepackage{pgfplots}
% \pgfplotsset{compat=1.15}

% Subfigures and wrapping text
% \usepackage{subcaption}

% Mathematics packages
\usepackage{amsmath, amssymb, amsthm}
%\usepackage{bm}

% Chemistry packages
% \usepackage{chemfig}
% \usepackage[version=4]{mhchem}

% Text hyperlinking
\hypersetup{hidelinks}

% (SI) unit handling
% \usepackage{siunitx}

% float placement
\usepackage{float}

% svg support
\usepackage{svg}

% landscape support
\usepackage{lscape}

% long table support
\usepackage{longtable}

% better wrapping
\usepackage[nopatch=footnote]{microtype}

% checkmark and x mark support from pifont's dingbats
\usepackage{pifont}
\newcommand{\cmark}{\ding{51}}%
\newcommand{\xmark}{\ding{55}}%

\usepackage{nowidow} 


%\sisetup{
%    detect-all,
%    math-sf=\mathrm,
%    exponent-product=\cdot,
%    output-decimal-marker={,} % for theses in FINNISH!
%}

%%%%% Your commands.

% Print verbatim LaTeX commands
\newcommand{\verbcommand}[1]{\texttt{\textbackslash #1}}

% Basic theorems in Finnish and in English.
% Remove [chapter] if you wish a simply
% running enumeration.
% \newtheorem{lause}{Lause}[chapter]
% \newtheorem{theorem}[lause]{Theorem}

% \newtheorem{apulause}[lause]{Apulause}
% \newtheorem{lemma}[lause]{Lemma}

% Use these versions for individually
% enumerated lemmas
% \newtheorem{apulause}{Apulause}[chapter]
% \newtheorem{lemma}{Lemma}[chapter]

% Definition style
% \theoremstyle{definition}
% \newtheorem{maaritelma}{Määritelmä}[chapter]
% \newtheorem{definition}[maaritelma]{Definition}
% examples in this style

%%%%% Glossary information.

% Use the following lines ONLY if you need more
% than one glossary. The first argument specifies
% a type label for the glossary and the second
% the displayed name.
% \newglossary*{symbs}{Symbols}
% \newglossary{label}{Displayed name}
% ...

\makeglossaries

% Use this line if using the default glossary.
% Otherwise comment out.
\loadglsentries[main]{tex/sanasto.tex}

% Use this line if using more than one glossary.
% Otherwise comment out.
% \loadglsentries[symbs]{tex/sanasto2.tex}

%%%%% Citation information.

% Commonly used bibliography modifications.
% Feel free to play around with them.

%\ExecuteBibliographyOptions{%
%sorting=none,
%maxbibnames=99,
%maxcitenames=2,
%giveninits=true,
%uniquename=init,
%sortcites,
%sortlocale=fin}

%\DefineBibliographyStrings{finnish}{%
%    in = {},
%    pages = {s.},
%    page = {s.}
%}
%\DefineBibliographyStrings{english}{%
%    in = {},
%    pages = {pp.},
%    page = {p.}
%}
%
%\DeclareNameAlias{sortname}{last-first}
%\DeclareNameAlias{author}{last-first}

%\DeclareFieldFormat[%
%    article,inbook,incollection,inproceedings,
%    patent,thesis,unpublished]{citetitle}{#1\isdot}
%\DeclareFieldFormat[%
%    article,inbook,incollection,inproceedings,
%    patent,thesis,unpublished]{title}{#1\isdot}
%\DeclareFieldFormat{pagetotal}{#1 \bibstring{page}}

% fix urldate format for Finnish
\DefineBibliographyStrings{finnish}{%
    urlseen = {Haettu\addspace},%
    urlfrom = {osoitteesta\addspace},%
}

\renewbibmacro*{url+urldate}{%
  \iffieldundef{urlyear}
    {}
    {\bibstring{urlseen}\usebibmacro{urldate}\addspace\bibstring{urlfrom}}%
     \usebibmacro{url}%
     }

\DefineBibliographyExtras{finnish}{%
 % d-m-y format for short dates
  \protected\def\mkbibdateshort#1#2#3{%
    \iffieldundef{#3}
      {}
      {\thefield{#3}%
       \iffieldundef{#2}{}{.}}%
    \iffieldundef{#2}
      {}
      {\thefield{#2}%
       \iffieldundef{#1}{}{.}}%
    \iffieldbibstring{#1}{\bibstring{\thefield{#1}}}{\mkdatezeros{\thefield{#1}}}}%
}

%\renewcommand*{\finentrypunct}{}

%\AtBeginBibliography{\renewcommand*{\makelabel}[1]{#1\hss}}

%\DefineBibliographyExtras{english}{\let\finalandcomma=\empty}

\addbibresource{tex/references.bib}

\begin{document}

%%%%% FRONT MATTER %%%%%

\frontmatter

%%%%% Thesis information and title page.

% The titles of the work. If there is no subtitle,
% leave the arguments empty. Pass the title in
% the primary language as the first argument
% and its translation to the secondary language
% as the second.
\title{Suorituskyvyntestaustyökalun valinta web-sovelluspalveluiden testaamiseen}{A Descriptive Title}
\subtitle{}{A Specifying Subtitle}

% The author name.
\author{Otto}

% The examiner information.
% If your work has multiple examiners, replace with
% \examiner[<label>]{<name> \\ <name>}
% where <label> is an appropriate (plural) label,
% e.g. Examiners or Tarkastajat, and <name>s are
% replaced by the examiner names, each on their
% separate line.
\examiner[Tarkastaja]{Petri}

% The finishing date of the thesis (YYYY-MM-DD).
\finishdate{2023}{05}{01}

% The type of the thesis (e.g. Kandidaatintyö
% or Master of Science Thesis) in the primary
% and the secondary languages of the thesis.
\thesistype{Kandidaatintutkielma}{Thesis type}

% The faculty and degree programme names in
% the primary and the secondary languages of
% the thesis.
\facultyname{Informaatioteknologian ja viestinnän tiedekunta
}{Faculty Name}
\programmename{Tieto- ja sähkötekniikan tutkinto-ohjelma}{Degree Programme}

% The keywords to the thesis in the primary and
% the secondary languages of the thesis
\keywords%
    {suorituskyky, suorituskyvyn testaus, suorituskyvyntestaustyökalu, suorituskyvyntestaustyökalun valinta, web-sovelluspalvelu}
    {keyword, keyword, keyword, keyword, keyword}

\maketitle

%%%%% Abstracts and preface.

% Write the abstract(s) and the preface
% into a separate file for the sake of clarity.
% Pass the appropriate file name as the first
% argument to these commands. Put the \abstract
% in the primary language first and the
% \otherabstract in the secondary language second.
% Those who do not speak Finnish only need the
% first abstract. The second argument of
% the \preface command takes the place where
% the thesis was signed in.
\abstract{tex/tiivistelma.tex}
%\preface{tex/alkusanat.tex}{Tampereella}


%%%%% Table of contents.

\tableofcontents

%%%%% Lists of figures, tables, listings and terms.

% Print the lists of figures and/or tables.
% Uncomment either of these commands as required.
% Both are optional, but if there are many important
% figures/tables, listing them may be a good idea.

% \listoffigures
% \listoftables
% \lstlistoflistings

% Misc stuff related to how the glossary is displayed.
% You can especially tweak the lengths to suit you!
\glsaddall
\setglossarystyle{taulong}
\setlength{\glsnamewidth}{0.25\textwidth}
\setlength{\glsdescwidth}{0.75\textwidth}
\renewcommand*{\glsgroupskip}{}

% Print the default glossary of abbreviations,
% if necessary. Otherwise comment out.
% The appropriate Finnish variant is 'Lyhenteet'
\printglossary[title={Lyhenteet ja merkinnät}]

% Print more than one glossary with these lines.
% Otherwise comment out.
% \printglossary[type=symbs]
% \printglossary[type=label]
% ...

%%%%% MAIN MATTER %%%%%

\mainmatter

% Write each of the chapters of the thesis
% into a separate file for the sake of clarity.
% They can be \input as shown below. Give both
% the chapters and their files as descriptive
% names as possible.
\chapter{Johdanto}
\label{ch:johdanto}
Yhä digitalisoituvassa nykymaailmassa tietoteknologiatuotteiden ja -palvelujen ala on biljoonien eurojen arvoinen. Hyvän IT-infrastruktuurin ylläpitäminen on välttämätöntä niin yrityksille, valtion palveluille kuin voittoa tavoittelemattomille järjestöillekin. \parencite{ITbudgets} Eri IT-järjestelmiä yhdistää web-sovelluspalvut (engl. web service), joten niiden moitteeton toimiminen on kriittinen osa IT-infrastruktuuria. Web-sovelluspalveluja kehitettäessä niiden moitteettomaan toimimiseen pyritään ohjelmistotestauksen avulla. Suorituskyvyntestaus on tärkeä osa ohjelmistotestausta, jolla varmistetaan muun muassa ohjelman nopea ja tehokas toimiminen. Suorituskykyä testataan suorituskyvyntestaustyökaluilla, joita on olemassa hyvin monia erilaisia. Kuitenkin sovelluskehitysprojektia varten täytyy ostata valita tarpeisiin sopiva suorituskyvyntestaustyökalu. 

Tässä työssä perehdytään tekijöihin, jotka vaikuttavat suorituskyvyntestaustyökalun valintaan. Suorituskyky, ja sen testaus, on kokonaisuudessaan todella laaja aihealue. Tämän kaiken käsittely ei mahtuisi tähän työhön. Tästä syystä työn näkökulma on rajattu web-sovelluspalveluiden suorituskyvyn testaamiseen työkaluilla. Tämä työ vastaa kysymyksiin, miten valitaan oikea suorituskyvyntestaustyökalu web-sovelluspalveluiden testaamiseen ja mitä tekijöitä pitää ottaa huomioon, kun vertaillaan eri työkaluja. Aiheesta löytyy materiaalia enimmäkseen oppaina. Oppaissa harvoin on mitään lähteitä ja ne perustuvat vain alan ihmisten omiin kokemuksiin ja mielipiteisiin. Tästä syystä on mielenkiintoista suorittaa aiheesta kirjallisuuskatsaus ja löytää tieteellisiä perusteita eri tekijöille, jotka vaikuttavat suorituskyvyntestaustyökalun valintaan.

Luvussa \ref{ch:tutkimusmenetelmä} kuvataan tutkimusmenetelmä ja tiedonhakuprosessi. Tiedonhaunprosessista kerrotaan käytetyt hakusanat ja millä perusteella lähteitä otettiin mukaan työhön. Luvussa \ref{ch:taustaselvitys} perehdytään työn aiheen taustatietoihin. Taustatietoihin kuuluu keskeisten käsitteiden ymmärtäminen, suorituskyvyntestauksen tärkeyden sekä toimintaperiaatteen hahmottaminen ja suorituskyvyntestaustyökalujen toimintaperiaatteiden käsittäminen. Luvussa \ref{ch:työkalunvalinta} esitetään työn tulokset. Tulokset ovat suorituskyvyntestaustyökalujen valintaan liittyvien tekijöiden muodossa. Monia eri tekijöitä kuvaillaan esimerkkien avulla ja niiden tärkeyttä perustellaan kirjallisuuskatsauksella löydettyjen lähteiden avulla. Luvussa \ref{ch:tulostentarkastelu} tarkastellaan saatuja tuloksia ja vertaillaan niitä internetistä löytyviin oppaisiin. Tarkastelussa esitetään kritiikkiä tuloksia kohtaan ja pohditaan työssä olleita rajoitteita. Lopuksi luvussa \ref{ch:yhteenveto} vedetään yhteen työn aihe ja saadut tulokset.


\chapter{Tutkimusmenetelmä}
\label{ch:tutkimusmenetelmä}
\input{tex/2.tutkimusmenetelmä.tex}

\chapter{Web-sovelluspalvelut ja suorituskyky}
\label{ch:taustaselvitys}
Tässä luvussa käydään läpi pohjatiedot työn aiheen ymmärtämiseksi. Ensin perehdytään web-sovelluspalveluihin. Tästä siirrytään suorituskyvyn ymmärtämiseen käsitteenä ja sen testaamiseen. Viimeiseksi kerrotaan taustatiedot itse suorituskyvyntestaustyökaluista.

\section{Web-sovelluspalvelut}
\label{sec:web-sovelluspalvelut}
Termi web-sovelluspalvelu, tai varsinkin sen arkikielinen nimitys web-palvelu, sekoitetaan usein muihin samankaltaisiin termeihin, kuten verkkopalveluihin. Termin määritelmä on kuitenkin \acrlong{w3c}in (\acrshort{w3c}) mukaan ohjelmistojärjestelmä, joka mahdollistaa yhteensopivien tietokoneiden vuorovaikutuksen jonkin tietokoneverkon yli. \parencite{w3c}

Web-sovelluspalveluilla on jokin rajapinta, joka mahdollistaa vuorovaikutuksen muiden järjestelmien kanssa. Vuorovaikutus tapahtuu käyttäen jotain protokollaa. Useimmissa tapauksissa protokollana toimii \acrshort{http} (engl. \acrlong{http}), mutta protokolla voi olla myös mikä tahansa muu. \parencite{w3c} Web-sovelluspalvelun tarjoama rajapinta, ja standardisoidun protokollan käyttö, mahdollistaa ohjelmointikielestä ja alustasta riippumattoman kommunikaation järjestelmien välillä. 

Muita esimerkkejä usein käytetyistä protokollista web-sovelluspalveluiden kanssa ovat esimerkiksi \acrshort{soap} (engl. \acrlong{soap}), \acrshort{rest} (engl. \acrlong{rest}), \acrshort{mqtt} (engl. \acrlong{mqtt}) ja \acrshort{coap} (engl. \acrlong{coap}). \acrshort{rest} ei ole käytännössä protokolla, vaan arkkitehtoninen tyyli toteuttaa sovellus. \acrshort{rest} kuitenkin rinnastetaan usein \acrshort{soap}:iin, koska ne ovat molemmat hyvin yleisiä tapoja toteuttaa web-sovelluspalvelu. Tästä syystä \acrshort{rest} esitellään tässä alaluvussa. \acrshort{mqtt} ja \acrshort{coap} ovat yleisiä \acrshort{iot}-protokollia (engl. \acrlong{iot}) ja ne esitellään tässä alaluvussa yhtenä esimerkkinä monista mahdollisista erilaisista protokollista.

\acrshort{soap} on vanhempi protokolla, joka luotiin ennen \acrshort{rest}:iä. \acrshort{soap} on \acrshort{rest}:iä hitaampi ja raskaampi, koska jokainen \acrshort{soap}-viesti sisältää siirrettävän datan lisäksi paljon oheistietoa \acrshort{xml}-formaatissa (engl. \acrlong{xml} format). Tämä oheistieto muun muassa kuvailee sovellusta, jossa protokolla on käytössä. \acrshort{rest}:iä noudattavat viestit sen sijaan ovat hyvin minimaalisia, ja eivät sisällä mitään oheistietoa siirettävän datan lisäksi. Tämä voi olla sekä hyöty, että haitta, riippuen halutusta käyttötarkoituksesta. \parencite{SOAPvsREST} \acrshort{mqtt}-protokolla toimii \acrshort{tcp}:n (engl. \acrlong{tcp}) päällä, kun taas \acrshort{coap} toimii \acrshort{udp}:n (engl. \acrlong{udp}) päällä \parencite{MQTTvsCoAP}. Nämä kaksi tunnettua \acrshort{iot}-protokollaa toimivat oivana esimerkkinä siitä, miten protokollat voivat hyödyntää toisiaan ja toimia toistensa päällä. Aivan kuten \acrshort{soap} voi toimia esimerkiksi \acrshort{http}:n tai jopa \acrshort{smtp}:n (engl. \acrlong{smtp}) päällä. \acrshort{smtp} on yleisin protokolla, joka mahdollistaa sähköpostien lähettämisen ja vastaanottamisen. 

\section{Suorituskyvyn mittaaminen}
\label{sec:suorituskyky}
Suorituskyky kertoo ohjelman nopeudesta ajan suhteen ja siitä, miten tehokkaasti ohjelma käyttää resursseja. Suorituskykyä ei ole määritelty jollain tietyllä yksiselitteisellä ominaisuuksien joukolla, mutta sitä voidaan kuvata tietyillä mittareilla. Yleisiä mittareita ovat reagointikyky, vasteaika, suoritusteho, saatavuus, skaalautuvuus ja käyttöaste. \parencites[4]{WhatIsSoftwarePerformance}[2]{PerformanceTestingGuidanceForWebApplications}[2-3]{TheArtOfApplication} Reagointikyky kertoo, miten nopeasti ohjelma pystyy reagoimaan tiettyyn asiaan. Vasteaika kertoo, kuinka kauan tietyllä tehtävällä kestää läpäistä ohjelma. Suoritusteho kertoo, miten paljon tiettyjä tehtäviä ohjelma pystyy suorittamaan ajan hetkessä. Saatavuus kertoo, kuinka suuren osuuden ajasta ohjelma on valmiina ottamaan uuden tehtävän eikä ole jumissa muun tehtävän kanssa. Skaalautuvuus kertoo ohjelman toimintakyvystä, kun käsiteltävän tehtävien tai datan määrää kasvatetaan. Käyttöaste kertoo, kuinka suuren osan ajasta tietty ohjelman osa tekee töitä verrattuna muihin ohjelman osiin.

Edellisessä kappaleessa esitettiin vain teknillinen näkökulma suorituskyvyn määritelmälle. Suorituskyvyn määritelmä riippuu näkökulmasta. Tavalliselle loppukäyttäjälle hyvä suorituskyky tarkoittaa sitä, että ohjelma suorittaa annetut tehtävät ilman havaittavaa hidastumista tai muuta ärsytystä \parencite[1-2]{TheArtOfApplication}. Tämän kautta päästään kiinni suorituskyvyn testaamisen vaikeuteen. Miten voidaan testata teknillisesti vaikeaa asiaa, jonka onnistuminen on loppujen lopuksi kiinni vain ihmisen havaitsemasta lopputuloksesta? 

\section{Suorituskyvyntestaus}
\label{sec:suorituskyvyntestaus}
Ohjelmistotestaus on tärkeä osa ohjelmiston kehityksen elinkaarta ja suorituskyvyntestaus on tärkeä osa ohjelmistotestausta \parencite{ScrutinyOnVariousApproaches}. Suorituskykyä testaamalla voidaan muun muassa arvioida tuotteen valmiutta, arvostella suorituskykyä haluttujen kriteerien perusteella, etsiä suorituskykyongelmien lähteitä tai arvioida tuotteen suorituskykyä eri järjestelmissä tai konfiguraatioissa \parencite{PerformanceTestingGuidanceForWebApplications}. Hyvä suorituskyky on myös tärkeää rahallisesti. Suuryritys Amazon löysi tutkimuksessa, että jokainen sadan millisekunnin viive tuotti yhden prosentin häviön vuotuisissa tuotoissa. Häviö tuotoissa perustui siihen, että käyttäjät eivät jaksaneet odottaa sivun latautumista, vaan sulkivat sen ja koettivat jotain muuta sivua hitaasti latautuvan sivun sijaan. \parencite{AmazonRevenue}

Toisin kuin monissa muissa testausmenetelmissä, suorituskyvyn testauksessa testataan ohjelman ei-toiminnallisia vaatimuksia. Luvussa \ref{sec:suorituskyky} mainittuja mittareita voidaan pitää ei-toiminnallisina vaatimuksia. Esimerkki toiminnallisen ja ei-toiminnallisen vaatimuksen erosta voisi olla viestintäpalvelu, jossa lähetetään viesti. Toiminnallinen vaatimus voisi olla, että viestin pystyy lähettämään ja vastaanottaja saa sen viestin. Ei-toiminnallinen vaatimus voisi olla, että järjestelmä pystyy skaalautumaan 10 000 yhtäaikaisen käyttäjän viestittelyyn.

Suorituskyvyntestaus on yleisnimitys monelle eri suorituskyvyntestaustyypille. Eri suorituskyvyntestaustyyppejä ovat muun muassa rasitustestaus (engl. stress testing), kuorman testaus (engl. load testing), vakaustestaus (engl. soak testing tai stability testing), skaalautuvuuden testaus (engl. scalability testing) ja piikkitestaus (engl. spike testing). Rasitustestauksessa rasitetaan ohjelmaa niin paljon, kunnes jokin menee rikki. Näin saadaan selville rajat, joissa ohjelma voi toimia. Kuorman testauksessa testataan, miten ohjelma toimii erikokoisten kuormien alaisena. Vakaustestauksessa testataan ohjelman vakautta ja toimintakykyä pitkäaikaisen suuren kuorman alaisena. Skaalautuvuuden testauksessa testataan ohjelman kykyä skaalautua kasvavaan käyttömäärään. Piikkitestauksessa testataan ohjelman kykyä suoriutua yllättävästä hetkellisestä suuresta kuormituksesta. \parencite{SoftwarePerformanceTesting, WebPerformanceTestingTools}

Kuvasta \ref{fig:agilequadrants} löytyvä Crispinin ja Gregoryn yleistämä ketterän testauksen neljän kvadrantin kuva vetää yhteen tässä alaluvussa käsitellyt asiat ja antaa oivan siirtymän suorituskyvyntestaustyökalujen käsittelyyn. Nelikentän oikea puoli, joka sisältää tuotetta kritisoivia testejä, liittyy vahvasti tämän työn aiheeseen. Oikealta puolelta löytyy monia tekijöitä, joita huomioidaan luvussa \ref{ch:työkalunvalinta}, kun pohditaan suorituskyvyntestaustyökalun valintaan liittyviä tekijöitä. Myös itse suorituskyvyntestaus löytyy nelikentän neljännestä kvadrantista, jonka kerrotaan sisältävän tuotetta kritisoivia ja teknologiaan keskittyviä testejä, joihin käytetään työkaluja. Nelikentän vasemmalla puolella olevat testit eivät suoraan liity tämän työn aiheeseen, koska ne ovat ohjelmistotestausprosessissa paljon aiempana kuin suorituskyvyntestaus.  

\begin{figure}[H]
\centering
\pdftooltip{\includesvg[width=.85\textwidth]{figures/kandi-testing-quadrants.svg}}
{Kuva. Neljä kvadranttia, joissa listataan ketterän kehityksen eri testausvaiheita. Suorituskykytestaus löytyy neljännestä kvadrantista}
\caption{Ketterän kehityksen testausprosessin neljä kvadranttia. Kuva mukailtu lähteestä \cite[97]{AgileTestingAPracticalGuide}.}
\label{fig:agilequadrants}
\end{figure}



\section{Suorituskyvyntestaustyökalut}
\label{sec:suorituskyvyntestaustyökalut}
Suorituskyvyntestaustyökalut ovat tietokonesovelluksia, joiden avulla voidaan suorittaa automatisoituja suorituskyvyntestejä. Suorituskyvyntestaustyökaluja on todella monia, ja niitä on ollut olemassa jo yli 30 vuotta. Eri työkaluja voidaan käyttää moniin erityyppisiin suorituskyvyntesteihin. Varsinkin web-sovelluksien testaamiseen työkaluja on monia. Kun siirrytään web-sovelluksien testauksesta muunlaisiin sovelluksiin, kuten esimerkiksi sulautettuihin järjestelmiin tai peleihin, sopivien suorituskyvyntestaustyökalujen määrä vähenee huomattavasti. \parencite[11-12]{TheArtOfApplication}

Suorituskyvyntestaustyökalujen tärkeimmät ominaisuudet koostuvat usein neljästä osasta, testien luonti, testien hallinta, testikuorman luonti ja tulosten analysointi \parencite[12]{TheArtOfApplication}. Eri työkaluissa nämä eri ydintoiminnot on toteutettu monilla eri tavoilla ja eri työkalut pystyvät erilaisiin toimintoihin. Testien luonti voi olla esimerkiksi koodin kirjoittamista, käyttöliittymässä kenttien täyttämistä tai toimintojen nauhoittamista. Testien hallinta sisältää testitapausten eri parametrien ja asetuksien säätämisen, sekä itse testien ajamisen. Testikuorman luontiin vaikuttaa työkalun kyky luoda erilaisia testikuormia. Esimerkiksi halutaan luoda kuorma, joka simuloi piikkitestausta, jossa 500 käyttäjää yrittää suorittaa jonkin toiminnon yhtä aikaa. Tulosten analysointiin liittyy toiminnot, joilla työkalu pystyy esittämään keräämänsä tulokset. Tähän voi liittyä erilaisia visualisointeja erilaisten diagrammien muodossa, joista käyttäjä voi tehdä itse johtopäätöksiä. \parencite{ComparativeAnalysisOfWeb} Näiden neljän ydintoiminnon lisäksi työkalut voivat sisältää myös muita moduuleja. Nämä perustoimintojen ulkopuolella olevat lisämoduulit usein täydentävät perustoimintojen ominaisuuksia. \parencite[13]{TheArtOfApplication} Lisämoduuli voisi olla esimerkiksi ohjelmointirajapinta (engl. \acrlong{api}, \acrshort{api}), jonka avulla työkalu voidaan liittää muihin järjestelmiin.

Perimmäinen syy työkalujen käyttöön testauksessa on testien automatisointi. Testien automatisoinnilla on monia perinteisiä hyvin tunnettuja etuja manuaaliseen testaukseen nähden, kuten esimerkiksi testauksen nopeuttaminen tai rahan säästäminen \parencite[528]{AgileTestingAPracticalGuide}. Automatisoinnin tuomien perinteisten hyötyjen lisäksi suorituskyvyntestauksessa on erityisiä hyötyjä työkalujen käytöstä. Useimmat luvussa \ref{sec:suorituskyky} mainituista mittareista ovat hyvin vaikeita, tai jopa mahdottomia, mitata manuaalisesti. Esimerkiksi manuaalinen skaalautuvuuden testaaminen olisi hyvin vaikeaa, kun haluttaisiin kasvattaa yhtäaikaista käyttäjien määrää suureksi. Suorituskyvyntestaustyökalu sen sijaan voi simuloida hyvinkin suuria määriä yhtäaikaisia käyttäjiä ongelmitta.


\chapter{Suorituskyvyntestaustyökalun valinta}
\label{ch:työkalunvalinta}
\input{tex/4.työkalunvalinta.tex}

\chapter{Tulosten tarkastelu}
\label{ch:tulostentarkastelu}
Tässä luvussa tarkastellaan löydettyjä tuloksia. Tuloksien pätevyyttä ja luotettavuutta pohditaan vertailemalla niitä aiempiin tuloksiin ja esittämällä kritiikkiä niitä kohtaan. Työn tulokset ovat erilaisten suorituskyvyntestaustyökalujen valintaan vaikuttavien tekijöiden muodossa. Tekijät, jotka löydettiin kirjallisuuskatsauksen avulla, löytyvät tämän työn tulosluvusta \ref{ch:työkalunvalinta}. Luvussa esitetyt tekijät jaettiin selkeyden vuoksi kahteen eri luokkaan. Ensimmäinen luokka oli \hyperref[sec:organisatorisettekijät]{organisatoriset tekijät} ja toinen luokka oli \hyperref[sec:työkalukohtaisettekijät]{työkalukohtaiset tekijät}. 

Organisatoristen tekijöiden alaluvussa löydetyt tekijät jaettiin kolmeen kategoriaan, \hyperref[ssec:hinta]{hintaan}, \hyperref[ssec:tarpeensuuruus]{työkalun tarpeen suuruuteen} ja \hyperref[ssec:tuki]{saatavilla olevaan tukeen}. Hintatekijöissä tietoa perusteltiin eri työkalujen hinnoittelusivuilta löytyviin tietoihin ja numeroihin. Kuitenkin monen työkalun hinnoittelusivussa luki myös mahdollisuus ottaa myyjään suoraan yhteys ja tehdä oma sopimus. Tätä ei otettu huomioon, eikä työssä ole tietoa mahdollisten omien sopimusten sisällöstä. Nämä tuntemattomat tekijät voisivat vaikuttaa joihinkin johtopäätöksiin, jotka tehtiin alaluvussa \ref{ssec:hinta}. Työkalun käytön tarvetta pohdittaessa käytettiin paljon omaa pohdintaa, jota ei ollut perusteltu lähteisiin. Tieto kuitenkin on luonnostaan yleispätevää ja sitä sivuttiin monissa eri lähteissä. Työkaluille saatavilla olevaa tukea käsiteltäessä esitettiin taulukko, jossa vertailtiin Stack Overflow -sivustosta löytyviä kysymysten määrää eri työkalujen välillä. Taulukko antoi selvät tulokset, joista tehtiin johtopäätöksiä. On syytä kuitenkin ottaa huomioon, että Stack Overflow on vain yksi sivusto. On mahdollista, että tietyistä työkaluista puhutaan hyvinkin paljon joissain muissa sivustoissa. 

Työkalukohtaisten tekijöiden alaluvussa löydetyt tekijät jaettiin myös kolmeen kategoriaan, testaustyyppeihin, yhteensopivuuteen ja käytettävyyteen. Tarvittavia testaustyyppejä tarkasteltaessa joitain päätelmiä tehtiin perustuen lähteisiin, jotka ovat yli 10 vuotta vanhoja. Tässä työssä on monia lähteitä, jotka eivät välttämättä edusta alan kaikista uusinta tietoa. Koska tämän työn tuloksia perustellaan lähteillä, eikä ajankohtaisilla henkilökohtaisilla alan ammattilaisen kokemuksilla, on syytä miettiä lähteiden pätevyyttä. Kuitenkin vanhemmista lähteistä löydetyt tiedot ovat usein yleispäteviä, joten on hyvin mahdollista, että tieto on yhä aiheellista. Yhteensopivuustekijöitä käsiteltäessä olisi hyvä olla joitain konkreettisia esimerkkejä siitä, miten suorituskyvyntestaustyökalujen toiminnallisuuksia voidaan laajentaa. Esimerkiksi olisi voitu kehittää oma lisämoduuli, jolla voisi demonstroida tietyn puuttuvan ominaisuuden lisäämistä työkaluun. Kuitenkin kokeellisen osuuden tuottaminen menee yli tämän työn tavoitteiden ja rajausten. Käytettävyyttä käsittelevä alaluku antaa myös esimerkin tässä työssä tehdyistä rajauksista, jotka estävät aiheen kasvamisen liian suureksi. Suorituskyvyntestaustyökalujen käytettävyydestä voisi tarkastella hyvinkin monta eri asiaa. Kuitenkin rajausten takia \hyperref[ssec:käytettävyys]{alaluvussa} päädyttiin käsittelemään vain työkalujen kahta ydintoimintoa, testien luontia ja tulosten analysointia.

Työn tuloksia voidaan käsitellä listana tekijöitä, joita pitää huomioida suorituskyvyntestaustyökalua valittaessa. Tällöin tämän työn tuloksia voidaan vertailla luvussa \ref{ch:johdanto} mainittuihin oppaisiin, joissa on muiden löytämiä listoja tekijöistä. Otetaan vertailuun kaksi Google-haulla löydettyä opasta, joita ei käytetty tämän työn tulosten tuottamiseen. \citeauthor{Tools8ThingsToConsider}in (\citeyear{Tools8ThingsToConsider}) ja \citeauthor{HowToChooseTheRight}in (\citeyear{HowToChooseTheRight}) oppaista löydetyt tekijät on koottu taulukkoon \ref{tab:vertailu}, jossa niitä vertaillaan tämän työn tuloksiin.

\begin{longtable}{|l|l|c|}
\caption{Nettioppaissa esiteltyjen valintatekijöiden löytyminen tästä työstä}
\label{tab:vertailu}\\
\hline
\multicolumn{1}{|l|}{\textbf{Nettiopas}} & 
\multicolumn{1}{|l|}{\textbf{Valintatekijä}} & 
\multicolumn{1}{l|}{\textbf{Tämä työ}} \\ \hline
\endfirsthead
%
\endhead
%
\citeauthor{HowToChooseTheRight, Tools8ThingsToConsider} & Protokollien tuki           & \hyperref[ssec:yhteensopivuus]{\cmark}                \\ \hline
\citeauthor{HowToChooseTheRight, Tools8ThingsToConsider} & Eri testaustyypit           & \hyperref[ssec:tarvittavatestaustyyppi]{\cmark}       \\ \hline
\citeauthor{HowToChooseTheRight} & Tulosten raportointi        & \hyperref[ssec:käytettävyys]{\cmark}                  \\ \hline
\citeauthor{HowToChooseTheRight, Tools8ThingsToConsider} & Lisenssimallit              & \hyperref[ssec:hinta]{\cmark}                         \\ \hline
\citeauthor{HowToChooseTheRight, Tools8ThingsToConsider} & Tuki                        & \hyperref[ssec:tuki]{\cmark}                          \\ \hline
\citeauthor{HowToChooseTheRight, Tools8ThingsToConsider} & CI/CD                       & \hyperref[ssec:yhteensopivuus]{\cmark}                \\ \hline
\citeauthor{HowToChooseTheRight, Tools8ThingsToConsider} & Yhteensopivuus              & \hyperref[ssec:yhteensopivuus]{\cmark}                \\ \hline
\citeauthor{Tools8ThingsToConsider} & Mukautettavuus              & \hyperref[ssec:yhteensopivuus]{\cmark}                \\ \hline
\citeauthor{HowToChooseTheRight, Tools8ThingsToConsider} & Hinta                       & \hyperref[ssec:hinta]{\cmark}                         \\ \hline
\citeauthor{HowToChooseTheRight} & Omien palvelimien hinta     & \hyperref[ssec:hinta]{\cmark}                         \\ \hline
\citeauthor{HowToChooseTheRight} & Osaaminen työmarkkinoilla   & \hyperref[ssec:tuki]{\cmark}                          \\ \hline
\citeauthor{HowToChooseTheRight} & Nauhoitusominaisuus         & \hyperref[ssec:käytettävyys]{\cmark}                  \\ \hline
\citeauthor{HowToChooseTheRight} & Parametrisointi             & \hyperref[ssec:käytettävyys]{\cmark}                  \\ \hline
\citeauthor{HowToChooseTheRight} & Käytettävyys                & \hyperref[ssec:käytettävyys]{\cmark}                  \\ \hline
\citeauthor{HowToChooseTheRight} & Tapahtuma \& pyyntö         & \xmark                                                \\ \hline
\citeauthor{HowToChooseTheRight} & Vastauksen vahvistaminen    & \xmark                                                \\ \hline
\citeauthor{HowToChooseTheRight} & Reaaliaikainen monitorointi & \hyperref[ssec:yhteensopivuus]{\cmark}                \\ \hline
\end{longtable}

Taulukossa näkyvän vertailun perusteella voidaan sanoa, että tämä työ onnistui hyvin kattavasti löytämään ja käsittelemään tekijöitä, joita oli myös muissa suorituskyvyntestaustyökalun valintaa käsittelevissä töissä. Puuttuvat tekijät eivät tulleet vastaan tämän työn kirjallisuuskatsauksessa ja jäivät sen takia täysin huomiotta. Tekijöiden puuttuminen tästä työstä voi johtua tämän työn rajauksista, kertoa vajaasta kirjallisuuskatsauksesta tai ehkä tekijät eivät vain ole kovin yleisiä ja ovat päässeet ainoastaan \citeauthor{HowToChooseTheRight}in henkilökohtaiselle listalle.

Ylipäätänsä tässä työssä jouduttiin asettamaan paljon rajoitteita, jotta aihe pysyy työn laajuustavoitteissa. Tiettyjä asioita käsiteltiin vain rajallisesti ja joitain asioita jätettiin pois heti aluksi, kuten tutkimusmenetelmää kuvaavassa luvussa \ref{ch:tutkimusmenetelmä} todettiin. Tästä huolimatta taulukossa \ref{tab:vertailu} tehdyn vertailun mukaan työn tulokset olivat hyvin kattavat. Näiden kaikkien päätelmien perusteella voidaan sanoa, että työn tulokset onnistuivat vastaamaan tämän työn tutkimuskysymyksiin.


\chapter{Yhteenveto}
\label{ch:yhteenveto}
Tässä työssä tarkasteltiin suorituskyvyntestaustyökalun valintaa web-sovelluspalveluille. Suorituskyvyntestaustyökaluja on hyvin monia, joten voi olla haastavaa vertailla eri työkaluja ja valita sopivin vaihtoehto. Asiaan perehdyttiin kirjallisuuskatsauksen keinoin ja pyrittiin löytämään vastaukset tutkimuskysymyksiin: 

\begin{enumerate}
    \item Miten valitaan oikea suorituskyvyntestaustyökalu web-sovelluspalveluiden testaamiseen?
    \item Mitä tekijöitä pitää ottaa huomioon, kun vertaillaan eri suorituskyvyntestaustyökaluja?
\end{enumerate}

Kirjallisuuskatsauksen perusteella suorituskyvyntestaustyökalujen valintaa pohditaan erilaisten tekijöiden avulla. Tästä syystä tämän työn ydintulokset koostuivat näiden tekijöiden löytämisestä ja analysoimisesta. Löydetyt tekijät jaettiin selkeyden vuoksi kahteen kategoriaan, organisatorisiin tekijöihin ja työkalukohtaisiin tekijöihin. Organisatoristen tekijöiden \hyperref[sec:organisatorisettekijät]{alaluvussa} perehdyttiin tekijöihin, jotka liittyivät suorituskyvyntestaustyökalujen \hyperref[ssec:hinta]{hinnoitteluun}, \hyperref[ssec:tarpeensuuruus]{tarpeen suuruuteen} ja \hyperref[ssec:tuki]{saatavilla olevaan tukeen}. Työkalukohtaisten tekijöiden \hyperref[sec:työkalukohtaisettekijät]{alaluvussa} perehdyttiin \hyperref[ssec:tarvittavatestaustyyppi]{tarvittavaan testaustyyppiin}, \hyperref[ssec:yhteensopivuus]{yhteensopivuuteen} ja \hyperref[ssec:käytettävyys]{käytettävyyteen}. Löydetyt tekijät, ja täten koko työn tulokset, olivat kattavat ja onnistuivat vastaamaan tämän työn tutkimuskysymyksiin hyvin. Työkalun valintaan vaikuttavia tekijöitä on kuitenkin monia, ja tästä syystä niitä kannattaa tarkastella tilanteen mukaan. Oma tarve saattaa vaatia jotain erikoistunutta ominaisuutta, jota ei mainita missään muiden tekemissä oppaissa. Näistä syistä voidaan todeta, että mitään yleispätevää ohjetta suorituskyvyntestaustyökalun valintaan ei voida mielekkäästi muotoilla.

Tässä työssä suorituskyvyntestaustyökalun valintaan liittyvä pohdinta oli rajattu web-so\-vel\-lus\-pal\-ve\-lui\-hin. Rajaus tehtiin, koska suorituskyvyntestaus, ja siihen käytettävät työkalut, on kokonaisuudessaan liian laaja aihe tämän työn käsiteltäväksi. Rajausten puitteissa karsittiin pois myös lähteitä, joissa työkaluja vertailtiin eri tekijöihin perehtymisen sijaan joillain täysin muilla lähestymistavoilla. Rajausten ulkopuolelle jätetyistä aiheista löytyy vielä mielenkiintoisia jatkotutkimusaiheita. Olisi mielenkiintoista esimerkiksi vertailla eräässä tutkimuksessa kehitettyä matemaattista suorituskyvyntestaustyökalun valintamallia perinteiseen tekijöiden pohtimiseen. Olisi myös mahdollista perehtyä täysin oman työkalun kehittämiseen ja vertailla sen toteuttamiskelpoisuutta ja mahdollisia kustannuksia valmiin työkalun käyttämiseen.
\looseness=-1



%%%%% Bibliography/references.

% Print the bibliography according to the
% information in ./tex/references.bib and
% the in-line citations used in the body of
% the thesis.
% \emergencystretch=2em

\printbibliography[heading=bibintoc]

%%%%% Appendices.

% Use only if it clarifies the structure of
% the document. Remember to introduce each
% appendix and its content.

%\begin{appendices}

%\chapter{Esimerkkiliite}
%\label{ch:liite}
%Tämä teksti toimii esimerkkinä liitteiden muodostamiseen tässä dokumenttipohjassa. Vähän pidempi saa siitä kokonaisen kappaleen näköisen.


%\end{appendices}

\end{document}